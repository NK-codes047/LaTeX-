\documentclass[11pt]{article}
\usepackage{graphicx}
\begin{document}
    \title{Machine Cycle in 8085 Microprocessor}
    \author{NK-codes047}
    \date{}
    \maketitle
    In this article the understanding of Machine cycle in 8085 is illustrated.
    \newline\\
    \begin{huge}
        Machine Cycle Diagram\\
    \end{huge}
    A general machine cycle diagram is shown bello:-\\
    \includegraphics[scale=0.4]{machine-cycle-diagram-1024x687-transformed.jpeg}
    \textbf{Instruction Cycle : }
        It is the total time taken by the Microprocessor to complete the execution of an instruction. The 8085 microprocessor contains generally about 1-5 machine cycles.\\
    \textbf{Machine Cycle : }
        It is the total time taken by the Microprocessor to complete any operation of accessing either memoryor I/O which is the sub-part of an instructon. The 8085 microprocessor contains generally about 3-6 T-state.\\
    \textbf{T-state : }
        It is termed as the subdivisin of an operation, that takes one clock period to complete.\\\\
    \begin{huge}
        Power Supply
    \end{huge}\\
    A 8085 Microprocessor has the following power supply specification :-\\
    \textbf{Vcc: }+5 Volt\\
    \textbf{Vss: }Ground or -ve terminal of power supply\\\\
    \begin{huge}
        Clock Frequency
    \end{huge}\\
    \textbf{X1,X2: }These are the pair crystal that is connected across the pins of the IC.
    The frequency is divided by 2 inside the IC. So in order to operate the system at 3 MHz, the crystal must be of frequency 6 MHz.\\
    \textbf{Clkout or clock output: }this singnal can be used as the system clock for other devices.\\\\
    \begin{huge}
        External singnals \\
    \end{huge}
    The 8085 has 5 hardeare interrupt singnals , as follows:- \\
    \textbf{TRAP: } This is a Nonmaskable interrupt has the highest priority.\\
    then according the priority comes the \textbf{RST 7.5, RST 6.5 and RST 5.5}\\
    \textbf{Restart Interruots: }These are vectored interrupts that transfer the program control to specific memory locations.\\
    \textbf{Maskable }interrupts having fixed memory locations i.e. 7.5 x 8 = 003CH, 6.5 x 8 = 0034H, 5.5 x 8 = 002CH \\
    the memory location for TRAP is 2.4 x 8 = 0024H\\
    More instructions are as followed:-\\
    \textbf{1. TNTR: }Interrupt Request - it has the lowest priority among all the interrupt.\\
    \textbf{2. INTA: }Interrupt Acknowledge - this is used to Acknowledge an interrupt.\\
    \textbf{3. HOLD: }This singnal indicates that a peripheral such as a DMA controller is requesting the use of the address and data buses.\\
    \textbf{4. HLDA: }Hold Acknowledge - this is singnal Acknowledge the HOLD request.\\
    \textbf{5. READY: }This singnal is used to delay the microprocessor Read and Write cycleuntil a slow responding peripheral is ready to send or acet data.\\
    \textbf{6. REST IN: }When the singnal on this pin goes low, the program controller is set to zero and the microprocessor is reset.\\
    \textbf{7. REST OUT: }This singnal indicates that the MPU is being rest and also reset other devices.\\
\\
\begin{huge}
    Details of Interrupts \\
\end{huge}
\begin{huge}

\end{huge}





\end{document}
